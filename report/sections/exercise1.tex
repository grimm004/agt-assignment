
Consider the following instance of the load balancing game where the number of tasks is equal to the number of machines, and in particular we have:
\begin{itemize}
    \item $m$ identical machines $M_1, M_2, \dots, M_m$ (all of speed 1),
    \item $m$ identical tasks $w_1 = w_2 = \dots = w_m = 1$.
\end{itemize}
Consider also the mixed strategy profile $A$ where each of the tasks is assigned to all machines equiprobably (i.e. with probability $1/m$).
\begin{enumerate}
    \item[(a)] Calculate the ratio $cost(A)/cost(OPT)$ in the special case where $m=2$.  \hfill{\bf [3 marks]}\smallskip

    Trivially, makespans of $1$ and $2$ have $2$ assignments each.
    Hence, for $m = 2$, there are a total of $2^2 = 4$ possible assignments.
    These assignments are shown in Table~\ref{tab:ex1a}.

    \begin{table}[ht!]
        \centering
        \begin{tabular}{cccc}
          & $M_1$ & $M_2$ & Makespan \\ \hline
        1 & 1, 2  & -     & 2        \\
        2 & -     & 1, 2  & 2        \\
        3 & 1     & 2     & 1        \\
        4 & 2     & 1     & 1
        \end{tabular}
        \caption{Task-machine assignments for $m = 2$ in Exercise 1}
        \label{tab:ex1a}
    \end{table}

    From this, $cost(A)$ can be calculated with
    \begin{equation}
        cost(A) = E[cost(B)] = \sum_{i = 1}^{m} P(cost(B) = i) \cdot i
        \label{eq:ex1-cost-a}
    \end{equation}

    Since there is capacity for one machine per task, and this would be the optimal assignment for any positive $m$, hence, it holds true that
    \begin{equation}
        \forall{m} > 0, cost(OPT) = 1
        \label{eq:ex1-opt-cost}
    \end{equation}

    For $m = 2$, using~\eqref{eq:ex1-cost-a}, $cost(A) = E[cost(B)] = \frac{1}{4} (1 \cdot 2 + 2 \cdot 2) = \frac{6}{4} = \frac{3}{2} = 1.5$.

    Combining this with~\eqref{eq:ex1-opt-cost}, the ratio $cost(A)/cost(OPT)$ for $m = 2$ is $\frac{3}{2} / 1 = \frac{3}{2} = 1.5$

    \item[(b)] Calculate the ratio $cost(A)/cost(OPT)$ in the special case where $m=3$.  \hfill{\bf [3 marks]}\smallskip

    For a makespan of $1$, there are $\cperm[3]{3} = 3! = 6$ assignments, for $2$, there are $3 \cdot \cperm[3]{2} = 18$ assignments and for $3$, there are trivially $3$ assignments.
    Hence, for $m = 3$ there are a total of $3^3 = 27$ possible assignments.
    These assignments are shown in Table~\ref{tab:ex1b}.

    \begin{table}[ht!]
        \centering
        \begin{tabular}{ccccc}
           & $M_1$   & $M_2$   & $M_3$   & Makespan \\ \hline
        1  & 1, 2, 3 & -       & -       & 3        \\
        2  & -       & 1, 2, 3 & -       & 3        \\
        3  & -       & -       & 1, 2, 3 & 3        \\
        4  & 1       & 2, 3    & -       & 2        \\
        5  & 1       & -       & 2, 3    & 2        \\
        6  & -       & 1       & 2, 3    & 2        \\
        7  & 2, 3    & 1       & -       & 2        \\
        8  & 2, 3    & -       & 1       & 2        \\
        9  & -       & 2, 3    & 1       & 2        \\
        10 & 2       & 1, 3    & -       & 2        \\
        11 & 2       & -       & 1, 3    & 2        \\
        12 & -       & 2       & 1, 3    & 2        \\
        13 & 1, 3    & 2       & -       & 2        \\
        14 & 1, 3    & -       & 2       & 2        \\
        15 & -       & 1, 3    & 2       & 2        \\
        16 & 3       & 1, 2    & -       & 2        \\
        17 & 3       & -       & 1, 2    & 2        \\
        18 & -       & 3       & 1, 2    & 2        \\
        19 & 1, 2    & 3       & -       & 2        \\
        20 & 1, 2    & -       & 3       & 2        \\
        21 & -       & 1, 2    & 3       & 2        \\
        22 & 1       & 2       & 3       & 1        \\
        23 & 2       & 1       & 3       & 1        \\
        24 & 2       & 3       & 1       & 1        \\
        25 & 3       & 2       & 1       & 1        \\
        26 & 3       & 1       & 2       & 1        \\
        27 & 1       & 3       & 2       & 1
        \end{tabular}
        \caption{Task-machine assignments for $m = 3$ in Exercise 1}
        \label{tab:ex1b}
    \end{table}

    For $m = 3$, using~\eqref{eq:ex1-cost-a}, $cost(A) = E[cost(B)] = \frac{1}{27} (1 \cdot 3 + 2 \cdot 18 + 3 \cdot 3) = \frac{51}{27} = \frac{17}{9} \approx 1.89$.
    Combining this with~\eqref{eq:ex1-opt-cost}, the ratio $cost(A)/cost(OPT)$ for $m = 3$ is $\frac{17}{9} / 1 = \frac{17}{9} \approx 1.89$

    \item[(c)] Discuss what this ratio is for arbitrary $m$. What does this imply about the Price of Anarchy on identical machines for mixed Nash equilibria?  \hfill{\bf [5 marks]}\smallskip

    There are $m^m$ possible assignments distributed over $m$ makespan values (from $1$ to $m$).

    As~\eqref{eq:ex1-opt-cost} holds true for all $m > 0$, the denominator of the fraction is always one.

\end{enumerate}
\vspace*{0.8cm}
